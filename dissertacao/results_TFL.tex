\chapter{Results}
\label{section:Results}

The reinforcement techniques presented in the sub section \ref{Reinforcement_Strategies} were used to replace the regular built-in fundamental components in more complex structures, with the purpose of enhancing their robustness. The referred structures are the classical NOT and 3-input majority logic gates from \cite{tougaw94}, as well as a full adder and a RCA adapted from \cite{bruschi11}.

Then, the modified structures were submitted to the error exploration methodology developed in this work, by means of the QCA Defects Simulator. Some results of the simulations with the original, modified and robust circuits already reported in the literature are shown and analyzed in the subsequent sections of this chapter.  

\section{Inverter}
\label{section:Inverter}

Three NOT gates were submitted to the QCA Defects Simulator under the four classes of structural defects, selected one at a time and combined. The defects were inserted into the devices by means of the uniform probability model. 
For simplification purposes, the structures are called here as INV1 (the original inverter from \cite{tougaw94}), INV2 (the \citeonline{tougaw94} inverter with modifications in the input, turnings and in the output) and INV3 (the robust inverter proposed by \citeonline{beard06}).

The remaining parameters of the QCA Defects Simulator were set as follows. The testing stop criteria required that the error-free rate variation did not surpass the value of 1\% along at least 100 iterations. Whether this criteria would not be achieved within 5000 iterations, the testing process should be interrupted (Maximum number of iterations parameter). However, such number of iterations has never been reached, which indicates that the testing process has converged as desired. Finally, the sampling interval was set to 10\% and the threshold for which a signal was interpreted as logic zero or one is 80\% of the standard polarization level limits (-1 and +1).

Table \ref{table:inverter} synthesizes the results obtained when the aforementioned NOT gates were submitted to random structural defects from all of the four possible defect classes with uniform probability. The results for the individual selection of defect classes are available in the Appendix \ref{Appendix_Simulation_Results}. 

\begin{table}[H]
\centering
\caption{The error-free simulations rate, the number of iterations required for the testing completion and the number of cells for three types of NOT gates under random defects from combined defect classes.}
\label{table:inverter}
\begin{tabular}{|c|c|c|c|}
\hline
 & Error-free (\%) & Iterations counter & Number of cells \\
\hline
 INV1 & 77.78 & 351 & 8 \\
\hline
 INV2 & 93.36 & 301 & 16 \\
\hline
 INV3 & 91.43 & 315 & 12 \\
\hline
\end{tabular}
\end{table}

The comparison between the quantitative values of the error-free simulations rates indicates that the modified inverter (INV2) surpasses the performance of both the original (INV1) and that proposed by \citeonline{beard06} (INV3) in 15.58\% and 1.93\% respectively. Such results endorse the efficiency of the reinforcement strategies utilized. Besides the numerical data, the heat maps created at the end of the tests are depicted in the Figure \ref{figure:inverter_t1}.

\begin{figure}[H]
\center
\subfloat[Regular inverter under combined defects.\label{figure:inverter_reg_t1}]{\includegraphics[width=0.45\textwidth]{images/SR_FC_51_INVERTER_REG_TEST1}
}
\hfill
\subfloat[Modified inverter under combined defects.\label{figure:inverter_mod1_t1}]{\includegraphics[width=0.45\textwidth]{images/SR_FC_52_INVERTER_MOD_TEST1}
}
\linebreak
\hfill
\subfloat[Beard inverter under combined defects.\label{figure:inverter_mod2_t1}]{\includegraphics[width=0.45\textwidth]{images/SR_FC_53_INVERTER_MOD2_TEST1}
}
\linebreak
{\includegraphics[width=0.8\textwidth]{images/LEGENDA}
}
\caption{Regular,modified and Beard inverter under combined defects}
\label{figure:inverter_t1}
\end{figure}

From the analysis of the heat maps it is possible to conclude that the structural modifications made to selective points allow the slighting of the color of the critical regions in at least one shade. Such fact may be observed for the INV2 input and output cells, where cells are added to the surroundings turn into light blue. Moreover, the color slighting is also noticed in one of the INV3 polarization inverter devices (the cell in the middle down-left).


\section{3-input Majority}
\label{section:Majority}

Three 3-input Majotity gates were submitted to the QCA Defects Simulator under the four classes of structural defects, selected one at a time and combined. The defects were inserted into the devices by means of the uniform probability model. 
For simplification purposes, the structures are called here as MAJ1 (a variation of the first QCA majority gate proposed in \cite{tougaw94}), MAJ2 (the same \citeonline{tougaw94} 3-input majority with modifications in the input, turnings and in the output) and MAJ3 (the robust 3-input majority block proposed by \citeonline{fijany01}).

The remaining parameters of the QCA Defects Simulator were set similarly to those reported for the inverters in sub section \label{section:Inverter}. Alike, the maximum number of 5000 iterations has never been reached, which indicates the convergence of the testing proceeded as desired.

Table \ref{table:majority} synthesizes the results obtained when the aforementioned 3-input gates were submitted to random structural defects from all of the four possible defect classes with uniform probability. The results for the individual selection of defect classes are available in the Appendix \ref{Appendix_Simulation_Results}. 

\begin{table}[h]
\centering
\caption{The error-free simulations rate, the number of iterations required for the testing completion and the number of cells for three types of 3-input majority gates under random defects from combined defect classes.}
\label{table:majority}
\begin{tabular}{|c|c|c|c|}
\hline
 & Error-free (\%) & Iterations counter & Number of cells \\
\hline
 MAJ1 & 62.92 & 418 & 14 \\
\hline
 MAJ2 & 80.75 & 348 & 39 \\
\hline
 MAJ3 & 91.99 & 312 & 89 \\
\hline
\end{tabular}
\end{table}

The comparison between the quantitative values of the error-free simulations rates indicates that the modified 3-input majority (MAJ2) surpasses the performance of the original gate (MAJ1) in an expressive percentage of 17.83\%. Nevertheless, MAJ2 performance is below the percentage obtained for MAJ3 in more than 10\%. Such result was already expected, since the MAJ2 may be entirely built-in a MAJ3 structure. Indeed, the \citeonline{fijany01} block majority has 50 cells more than the proposed MAJ2, thus more redundancy mechanisms to avoid errors propagation. Besides the numerical data, the heat maps created at the end of the tests are depicted in the Figure \ref{figure:majority_t1}.

\begin{figure}[H]
\center
\subfloat[Regular majority under combined defects.\label{figure:majority_reg_t1}]{\includegraphics[width=0.45\textwidth]{images/SR_FC_63_MAJORITY_REG_TEST1}
}
\hfill
\subfloat[Modified majority under combined defects.\label{figure:majority_mod_t1}]{\includegraphics[width=0.45\textwidth]{images/SR_FC_64_MAJORITY_MOD_TEST1}
}
\linebreak
\hfill
\subfloat[Fijany et al majority under combined defects.\label{figure:inverter_mod2_t1}]{\includegraphics[width=0.45\textwidth]{images/SR_FC_65_MAJORITY_MOD2_TEST1}
}
\linebreak
{\includegraphics[width=0.8\textwidth]{images/LEGENDA}
}
\caption{Regular and modified majority under combined defects}
\label{figure:majority_t1}
\end{figure}

From the analysis of the heat maps it is possible to conclude that the structural modifications made to selective points allow the slighting of the color of the critical regions when comparing the MAJ1 and MAJ2 structures in at least one shade. Such fact may be observed for the MAJ2 input and output cells, where cells are added to the surroundings turn from yellow into green (1 shade). Moreover, the turning region slighted its color in 2 shades, turning from orange to light green.

Although few critical points may be observed in MAJ3, 3 cells are colored red: one of them next to the output and the remaining two next to the inputs. The color red indicates that when the cell turns into defective, error events occur in more than 99\% of the times. Thus, this color is highly undesirable to appear at a heat map of any robust QCA structure.

\section{Full Adder}
\label{section:Full_Adder}

Three types of full adders were submitted to the QCA Defects Simulator under the four classes of structural defects, selected one at a time and combined. The defects were inserted into the devices by means of the uniform probability model. 
For simplification purposes, the structures are called here as ADD1 (a variation of the full adder proposed in \cite{bruschi11}), ADD2 (the same \citeonline{bruschi11} full adder with modifications in the inputs, turnings and in the outputs) and ADD3 (the ultra-compact robust full adder from \cite{roohi15}).

The remaining parameters of the QCA Defects Simulator were set similarly to those reported for the inverters in sub sections \ref{section:Inverter}, \ref{section:Majority} and \ref{section:Full_Adder}. Alike, the maximum number of 5000 iterations has never been reached, which indicates the convergence of the testing process occurred as desired.

Table \ref{table:fulladder} synthesizes the results obtained when the aforementioned RCA were submitted to random structural defects from all of the four possible defect classes with uniform probability. The results for the individual selection of defect classes are available in the Appendix \ref{Appendix_Simulation_Results}. 

\begin{table}[h]
\centering
\caption{The error-free simulations rate, the number of iterations required for the testing completion and the number of cells for three different full adder implementations under random defects from combined defect classes.}
\label{table:fulladder}
\begin{tabular}{|c|c|c|c|}
\hline
 & Error-free (\%) & Iterations counter & Number of cells \\
\hline
 ADD1 & 70.87 & 381 & 108 \\
\hline
 ADD2 & 76.70 & 359 & 23 \\
\hline
 ADD3 & 64.23 & 420 & 205 \\
\hline

\end{tabular}
\end{table}

The comparison between the quantitative values of the error-free simulations rates indicates that the modified full adder (ADD2) surpasses the performance of both the original (ADD1) and that ultra-compact proposed by \citeonline{roohi15} (ADD3) in 5.83\% and 12.47\% respectively. Such results endorse the efficiency of the reinforcement strategies utilized. It is also interesting to highlight that the great difference between the error-free simulation rates of ADD2 and ADD3 - both originally considered as robust circuits - is not unexpected at all. When the ultra-compact robust ADD3 was proposed, it was tested only against single misalignment defects as reported in \cite{roohi15}. It seems that its ultra-compact feature represents a big concern when more than one defect classes are considered in a single analysis or when more than one device is defective at a time, a possible condition in the testing methodology proposed. 

Besides the error-free simulations rate presented by means of the Table \ref{table:fulladder}, the heat maps created at the end of the testing process are depicted in the Figure \ref{figure:full_t1}.

\begin{figure}[H]
\center
\subfloat[Regular full adder under combined defects.\label{figure:full_mod1_t1}]{\includegraphics[width=0.45\textwidth]{images/SR_FC_75_FULL_REG_TEST1}
}
\hfill
\subfloat[Roohi et al full adder under combined defects.\label{figure:full_mod2_t1}]{\includegraphics[width=0.45\textwidth]{images/SR_FC_77_FULL_MOD2_TEST1}
}
\hfill
\subfloat[Modified full adder under combined defects.\label{figure:full_mod2_t1}]{\includegraphics[width=0.8\textwidth]{images/SR_FC_76_FULL_MOD_TEST1}
}
\linebreak
{\includegraphics[width=0.8\textwidth]{images/LEGENDA}
}
\caption{Regular,Roohi et al and modified full adder under combined defects}
\label{figure:full_t1}
\end{figure}

Analyzing the full adders heat maps is a little complicated task, due to the high number of built-in fundamental components within the ADD1 and ADD2 structures. However, despite some cells next to the key points remain into red, it is possible to observe a general trend of slighting of the color in almost all regions of the structure. For instance, through visual analysis, it is noticeable that many green cells have turned into light blue, especially in the wires.

Regarding to ADD3, it is not possible to clearly distinct between critical/ non-critical regions. Therefore, all the regions of the structure has shown a worrisome number of critical cells under the defects testing methodology presented. It seems like it is a tradeoff between compactness and robustness in this case.

\section{4-bit Ripple-carry Adders}

Three types of 4-bit ripple-carry adders were submitted to the QCA Defects Simulator under the four classes of structural defects, selected combined together. The defects were inserted into the devices by means of the uniform probability model. 
For simplification purposes, the structures are called here as RCA1 (a variation of the RCA proposed in \cite{bruschi11}), RCA2 (the same \citeonline{bruschi11} RCA with modifications in the inputs, turnings and in the outputs) and RCA3 (the ultra-compact robust RCA from \cite{roohi15}).

The remaining parameters of the QCA Defects Simulator were set similarly to those reported for the inverters, majorities and full adders in sub sections \ref{section:Inverter}, \ref{section:Majority} and \ref{section:Full_Adder}. Alike, the maximum number of 5000 iterations has never been reached, which indicates the convergence of the testing occurred as desired.

Table \ref{table:RCA} synthesizes the results obtained when the aforementioned full adders were submitted to random structural defects from all of the four possible defect classes with uniform probability.

\begin{table}[H]
\centering
\caption{The error-free simulations rate, the number of iterations required for the testing completion and the number of cells for three different 4-bit RCA implementations under random defects from combined defect classes.}
\label{table:RCA}
\begin{tabular}{|c|c|c|c|}
\hline
 & Error-free (\%) & Iterations counter & Number of cells \\
\hline
 RCA1 & 72.27 & 299 & 519 \\
\hline
 RCA2 & 80.53 & 230 & 909 \\
\hline
 RCA3 & 64.93 & 223 & 169 \\
\hline

\end{tabular}
\end{table}

As expected, the analysis of the 4-bit RCAs proceeds a quite similar to that one reported in Section \ref{section:Full_Adder}), since they were implemented as systems derived from the union of four basic full adder blocks. Thus, the comparison between the quantitative values of the error-free simulations rates indicates that the modified 4-bit RCA (RCA2) surpasses the performance of both the original (RCA1) and that ultra-compact proposed by \citeonline{roohi15} (RCA3) in 8.26\% and 15.60\% respectively.

Likewise already concluded for the full adders ADD2 and ADD3, the great difference between the error-free simulation rates of RCA2 and RCA3 - both designed to be robust - is not unexpected at all, for the same reasons mentioned in the Section \ref{section:Full_Adder}.

Besides the error-free simulations rate presented by means of the Table \ref{table:RCA}, the heat maps created at the end of the testing process are depicted in the Figure \ref{figure:RCA_t1}.

\begin{figure}[H]
\center
\subfloat[Regular 4-Bit RCA under combined defects.\label{figure:RCA_mod1_t1}]{\includegraphics[width=0.85\textwidth]{images/SR_FC_87_RCA_REG_TEST1}
}
\linebreak
\subfloat[Modified 4-Bit RCA under combined defects.\label{figure:RCA_mod1_t1}]{\includegraphics[width=0.85\textwidth]{images/SR_FC_88_RCA_MOD_TEST1}
}
\linebreak
\subfloat[Roohi et al 4-Bit RCA under combined defects.\label{figure:RCA_mod2_t1}]{\includegraphics[width=0.3\textwidth]{images/SR_FC_88B_RCA_MOD_TEST1}
}
\linebreak
{\includegraphics[width=0.8\textwidth]{images/LEGENDA}
}
\caption{Regular,Roohi et al and modified 4-Bit RCA under combined defects}
\label{figure:RCA_t1}
\end{figure}

Highlighting the contribution of every component to the overall robustness of the system through a heat map becomes a non-trivial task for large structures as the 4-bit RCAs presented. For such reason, the analysis proceeds in more general terms. Despite some cells next to the key points remain into red, it is possible to observe a general trend of slighting of the color in almost all regions of the structure. For instance, through visual analysis, it is noticeable that many red cells have turned into light blue.

Finally, it is not possible to clearly distinct between critical/ non-critical regions in RCA3, likewise to its corresponding full adder block whose testing results were already presented in the Section \ref{section:Full_Adder}. Therefore, all the regions of the structure has shown a worrisome number of critical cells under the defects testing methodology presented. It seems like the tradeoff between compactness and robustness comes up again.
