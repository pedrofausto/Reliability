\chapter{Introdução}
\epigraph{\textit{Peace is not an absence of war, it is a virtue, a state of mind, a disposition for benevolence, confidence, justice.}}{-Baruch Spinoza}
\section{Motivação}
Sistemas computacionais exercem hoje uma tarefa de sumária importância em diversas área de atuação. O poder computacional tem facilitado a solução de problemas antes considerados de longa ou difícil obtenção. Para este intento, a integração, e consequentemente miniaturização, em escalas cada vez mais reduzidas foi necessária, aumentando a quantidade de cálculos que poderiam ser realizados em um menor intervalo de tempo \cite{Moore1965}.

Como consequência, a quantidade de efeitos indesejados, antes irrisórios, aumentou dramaticamente. Diversos destes causam a indisponibilidade temporária ou permanente de um circuito. Outros podem interferir de forma não-destrutiva na performance deste circuito, tornando-o não confiável.
Essa perda de confiabilidade pode ser atribuída, em alguns casos, ao envelhecimento dos dispositivos ao longo de seu uso, seja em condições normais de operação ou não, e que degradam sua performance.

Muitos esforços têm sido realizados na análise dos efeitos que causam este envelhecimento, bem como propostas para detecção e correção, de forma dinâmica, deste comportamento indesejado. Este trabalho de pesquisa visa descobrir através de simulações, e de seus dados obtidos, padrões de comportamento de circuitos integrados que estão gradativamente sendo degradados e propor uma metodologia que dê suporte às promissoras técnicas de verificação, avaliação e atuação contra falhas.
\section{Roteiro da dissertação}
Esta dissertação é organizada em 6 capítulos, incluindo esta introdução. Os capítulos restantes são:

Capítulo 2, Base Teórica, explicando de forma sucinta as bases sobre às quais este trabalho se sustenta. Discorre sobre o conceito geral de confiabilidade, bem como os efeitos e modelos utilizados no estudo da confiabilidade de circuitos eletrônicos.

Capítulo 3, Metodologia, apresenta o método proposto para solução do problema em mãos, além de introduzir conceitos pertinentes para o entendimento da mesma.

Capítulo 4, Fluxo de envelhecimento de circuitos integrados, descreve os passos necessários para envelhecer os dispositivos aqui empregados, desde a sua síntese até a extração dos resultados.

Capítulo 5, Resultados, discorre sobre quais saídas foram obtidas do fluxo, as métricas utilizadas para ajuizar a eficiência dos modelos, discute os resultados e realiza uma autocrítica à metodologia. 

Capítulo 6, Conclusões, encerra o trabalho analisando a contribuição da metodologia, as estimativas, os resultados e quais são melhorias futuras para o mesmo, seus pontos fracos e fortes.