\begin{resumo}
\begin{otherlanguage*}{brazil}
O estudo da confiabilidade de circuitos integrados tem se tornado de súbita importância para o entendimento, detecção e correção de falhas dos mesmos. A compreensão de como ou sob quais condições este envelhecimento se torna prejudicial a um sistema permitirá tomar decisões que sanem ou evitem estas condições.

Este trabalho estabelece um fluxo para análise e simulação de sistemas integrados que permite entender o envelhecimento dos  mesmos, em diferentes condições. Além disso, permite extrair e analisar dados que são relevantes para prever a falha dos mesmos e sirvam de entrada para sistemas de verificação, avaliação e atuação contra falhas utilizando-se de técnicas de aprendizado de máquina.

A metodologia desenvolvida permite a integração de técnicas de coleta de dados \textit{offline} e \textit{online} para atualização dos métodos de estimativa, além de permitir que novos sejam adicionados. O trabalho utiliza três métodos diferentes para prever o Tempo Médio para Falha (\textit{i.e.} Mean Time To Failure, MTTF) e o Tempo de Vida Restante para alguns circuitos de teste ISCAS-85 e uma cadeia de inversores.
O MTTF é estimado para cada um deles utilizando um Modelo Linear Generalizado (especificamente uma Regressão de Mínimos Quadrados Parciais), a Distância Euclideana e a Correlação de Pearson como métodos de predição.

Os resultados obtidos indicam que a representação das condições de operação dos sistemas através de perfis dinâmicos é mais realística do que a representação através de um perfil de operação que não varia no tempo, além de mais precisa. Adicionalmente, a predição do MTTF foi de aproximadamente 90\% para um modelo de Regressão de Mínimos Quadrados Parciais e de Distância Euclideana.   

\textbf{Palavras-chave}: Tempo de vida restante. Confiabilidade. Envelhecimento de circuitos integrados. Regressão de Mínimos Quadrados Parciais. Modelo Linear Generalizado. Distância Euclideana. Correlação de Pearson.
\end{otherlanguage*}
\end{resumo}
