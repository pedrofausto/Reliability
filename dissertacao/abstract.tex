\begin{resumo}[Abstract]
\begin{otherlanguage*}{english}
The study of integrated circuits reliability has become of sudden importance to the understanding, detection and correction of their failures. Compreend how or under which conditions this aging becomes harmful to a system will enable decisions to be made that will mitigate or prevent these conditions.

This work establishes a flow for analysis and simulation of integrated systems that allows us to understand the aging of it in different conditions. In addition, it allows us to extract and analyze data that are relevant to predict their failure and also serve as input to verification, evaluation and fault-tolerance systems using machine learning techniques.

The methodology developed allows the integration of offline and online data collection techniques to update estimation methods, as well as allowing new ones to be added. This work uses three different methods to predict the Mean Time To Failure and the Remaining Useful Lifetime for some ISCAS-85 test circuits and a inverter chain.
The MTTF is estimated for each of them using a Generalized Linear Model (specifically a Partial Least Squares Regression), Euclidean Distance and Pearson's Correlation as prediction methods.

The obtained results indicate that the representation of the operating conditions of the systems through dynamic profiles is more realistic than the representation through a operation profile that does not vary in time, and more precise. Additionally, the MTTF prediction was approximately 90\% for Partial Least Squares Regression and Euclidean Distance models.

\textbf{Keywords}: Remaing Useful Lifetime. Reliability. Aging of Integrated Circuits. Partial Least Squares. Generalized Linear Models. Euclidean Distance. Pearson Correlation.
\end{otherlanguage*}
\end{resumo}





