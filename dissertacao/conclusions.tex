\chapter{Conclusões}
Estimar o tempo de vida de sistemas integrados é imperativo diante do atual avanço tecnológico e do consequente estresse submetido a eles. Efeitos físicos antes ignorados, agora passam a contribuir de forma preponderante para a degradação de circuitos integrados.

Nesse contexto surge a necessidade de observar, investigar e adequar a operação de um sistema afim de mitigar ou reduzir o desgaste decorrente de um perfil de operação. Inúmeras técnicas foram desenvolvidas e estão disponíveis na literatura que permitem coletar informações de diferentes origens e interpretá-las.

A metodologia aqui proposta permite integrar a coleta destes dados, ao mesmo tempo que representa estas informações na forma de perfis de operações. Não apenas isso, mas a metodologia está pronta para simular, degradar e estimar o tempo de vida restante de sistemas.

Apresentando uma abordagem que independe da origem dos dados e das técnicas já existentes, este trabalho permite que o projetista adeque o fluxo às suas necessidades; seja otimizando os métodos preditivos aqui propostos, seja utilizando outros inteiramente diferentes. Além disso, o trabalho independe da tecnologia utilizada no sensoriamento dos parâmetros de operação, do nó tecnológico ou até mesmo da variação sofrida no processo de fabricação dos dispositivos.

Circuitos de teste foram degradados como prova de conceito, mostrando que modelos preditivos menos complexos são capazes de estimar o MTTF com uma precisão de aproximadamente 90\% no melhor dos casos. Em adição, um banco de dados que represente estaticamente as condições ambientais de operação de um sistema não se mostra realístico.

\section{Futuros trabalhos}
Trabalhos futuros envolvem a automatização completa de todo o processo de extração de caminhos críticos, síntese e representação esquemática.

Abordagens diferentes de aprendizado de máquina (\textit{p.ex.} redes neurais), podem ser utilizadas e comparadas aos métodos utilizados, averiguando o impacto na área, consumo e precisão das estimativas. Além disso, a metodologia pode ser melhorada para que seja possível modelar o BDPE em função do número de amostras e métodos de estimativa utilizados, permitindo ao projetista analisar o \textit{trade-off} entre precisão, área ocupada pelo BDPE e métodos escolhidos.
